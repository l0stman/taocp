\documentclass[a4paper,12pt]{article}
\newcommand{\newpar}[1]{\bigskip \noindent \textbf{#1.}}
\newcommand{\subpar}[1]{\medskip \noindent (#1)}
\newcommand{\la}{\leftarrow}
\newcommand{\ra}{\rightarrow}
\begin{document}

\newpar{1} $\lfloor 1.1\rfloor = 1$, $\lfloor -1.1\rfloor = -2$,
$\lceil -1.1\rceil = -1$, $\lfloor 0.9999\rfloor = 0$, $\lfloor \lg
35\rfloor = 5$.

\newpar{2} $\lceil \lfloor x\rfloor \rceil = \lfloor x\rfloor$.

\newpar{3}
\subpar{a} Suppose $x < n$.  Thus

\[ \lfloor x\rfloor \le x < n.\]

Reciprocally, suppose $\lfloor x \rfloor < n$.  Thus

\[ x-1 < \lfloor x \rfloor \le n-1.\]

So finally $x < n$.

\medskip
\subpar{b} Suppose $n \le \lfloor x\rfloor$.  Thus

\[ n \le \lfloor x \rfloor \le x.\]

Reciprocally, suppose $n \le x$.  Thus

\[ n \le x < \lfloor x \rfloor + 1.\]

Since the leftmost and rightmost members in the above inequalities are
integers, we deduce that $n \le \lfloor x\rfloor$.

\subpar{c} Suppose $\lceil x\rceil \le n$.  Thus,

\[ x \le \lceil x\rceil \le n.\]

Reciprocally, suppose $x\le n$.  Thus,

\[ \lceil x\rceil - 1 < x \le n.\]

Since the leftmost and rightmost members of inequalities are integers,
we deduce that $\lceil x\rceil \le n$.

\medskip
\subpar{d} Suppose $n < \lceil x\rceil$.  Thus,

\[ n \le \lceil x\rceil - 1 < x.\]

Reciprocally suppose $n < x$.  Thus,

\[ n < x \le \lceil x\rceil.\]

\medskip
\subpar{e}  The second equivalence is trivial.  For the first one,
$\lfloor x\rfloor = n$ if and only if $n$ is the greatest integer less
than or equal to $x$, that $n$ is less than $x$ and $n+1$ is greater
than $x$.  Reciprocally, let $n$ be an integer such that

\[ n \le x < n+1.\]

Thus by definition, we have $n \le \lfloor x\rfloor \le x < n+1$.  Or
rewriting the inequalities:

\[ 0 \le \lfloor x\rfloor - n < 1.\]

Since $\lfloor x\rfloor - n$ is an integer, we deduce that it's equal
to zero.

\medskip
\subpar{f} We could apply the same reasoning as in (e).

\newpar{4}  From (e), we have

\begin{eqnarray*}
  \lfloor -x\rfloor \le &-x& < \lfloor -x\rfloor + 1 \\
  -\lfloor -x\rfloor - 1 < &x& \le - \lfloor -x\rfloor
\end{eqnarray*}

We then deduce from (f) that $\lceil x\rceil = -\lfloor -x\rfloor$.

\newpar{5} For $x$ a positive real number, note:

\[ r(x) = \left\lfloor x + \frac{1}{2}\right\rfloor.\]

We have

\begin{eqnarray*}
  r(x) &=& \left\lfloor \lfloor x\rfloor + x \bmod{1} +
  \frac{1}{2}\right\rfloor \\
  &=& \lfloor x\rfloor + \left\lfloor x \bmod 1 +
  \frac{1}{2}\right\rfloor \\
  &=& \left\{
    \begin{array}{l}
      \lfloor x \rfloor\ \mbox{if $x \bmod 1 < \frac{1}{2}$} \\
      \lceil x \rceil\ \mbox{if $x \bmod 1\ge \frac{1}{2}$}
    \end{array}
    \right.
\end{eqnarray*}

If $x < 0$, then we have

\[
r(x) = \left\{
\begin{array}{l}
  \lfloor x \rfloor\ \mbox{if $x \bmod 1 \ge -\frac{1}{2}$}\\
  \lfloor x \rfloor - 1\ \mbox{if $x \bmod 1 < -\frac{1}{2}$}
\end{array}
\right.
\]

\newpar{6}  Note: $n = \lfloor \sqrt x\rfloor$.  By using the
properties in \textbf{3}, we have

\begin{eqnarray*}
  n^2 \le & x & < (n+1)^2 \\
  n^2 \le & \lfloor x \rfloor & \le x < (n+1)^2 \\
  n \le & \sqrt{\lfloor x\rfloor} & \le \sqrt{x} < n+1
\end{eqnarray*}

So finally, we deduce that $n = \lfloor\sqrt{x}\rfloor =
\left\lfloor\sqrt{\lfloor x\rfloor}\right\rfloor$.  Similarly if we
note $n = \lceil \sqrt{x}\rceil$, we have

\begin{eqnarray*}
  (n-1)^2 < &x& \le n^2 \\
  (n-1)^2 < &x& \le \lceil x\rceil \le n^2 \\
  n-1 < &\sqrt{x}& \le \sqrt{\lceil x\rceil}\le n^2
\end{eqnarray*}

Thus, finally $n = \left\lceil \sqrt{x}\right\rceil = \left\lceil
\sqrt{\lceil x\rceil}\right\rceil$.

On the other hand the equality (c) is false since

\[ \left\lceil \sqrt{\lfloor 1.5 \rfloor}\right\rceil = 1 < 2 =
\left\lceil \sqrt{1.5}\right\rceil.\]

\newpar{7}  We have easily the inequality

\[ \lfloor x \rfloor + \lfloor y\rfloor \le x + y.\]

Thus $\lfloor x\rfloor + \lfloor y\rfloor \le \lfloor x + y\rfloor$.
Moreover,

\begin{eqnarray*}
  \lfloor x\rfloor + \lfloor y \rfloor &=& x + y - (x \bmod 1 + y
  \bmod 1) \\
  &<& \lfloor x + y\rfloor + 1 - (x \bmod 1 + y \bmod 1)
\end{eqnarray*}

Thus, if $x \bmod 1 + y \bmod 1 \ge 1$ then

\[ \lfloor x\rfloor + \lfloor y\rfloor < \lfloor x + y\rfloor.\]

On the other hand, if $x \bmod 1 + y \bmod 1 < 1$, then

\begin{eqnarray*}
  \lfloor x\rfloor + \lfloor y\rfloor &=& x + y - (x\bmod1 +
  y\bmod1)\\
  &>& x + y - 1 \\
  &\ge& \lfloor x + y \rfloor -1
\end{eqnarray*}

We then deduce that $\lfloor x\rfloor + \lfloor y\rfloor > \lfloor x +
y\rfloor - 1$.  Thus $\lfloor x\rfloor + \lfloor y\rfloor \ge \lfloor
x + y\rfloor$, and finally we deduce the equality since the
reciproc is always true.

We have also $\lceil x + y\rceil \le \lceil x\rceil + \lceil y \rceil$
and there's equality if and only if $x$ or $y$ is an integer or

\[ x \bmod 1 + y \bmod 1 > 1.\]

\newpar{8} $100 \bmod 3 = 1$, $100 \bmod 7 = 2$, $-100 \bmod 7 = 5$,
$-100 \bmod 0 = -100$.

\newpar{9} $5 \bmod -3 = -1$, $18 \bmod -3 = 0$, $-2 \bmod -3 = -2$.

\newpar{10} $1.1 \bmod 1 = 0.1$, $0.11 \bmod .1 = 0.01$, $0.11 \bmod
-.1 = - 0.09$.

\newpar{11} It means that $x = y$.

\newpar{12} All integers.

\newpar{13} $1$ and $-1$.

\newpar{14} Let $p$ and $q$ be two integers such that $x = 3 p + 2$
and $x = 5 q + 3$.  We then deduce that $3p = 5q + 1$. Given that
$3\times 3 = 5 \times 2 - 1$, adding the two equalities we obtain:

\[ 3(p+3) = 5(q+2) .\]

Since $3\,\bot\,5 = 1$, we then deduce that $3$ divides
$q+2$. So finally, we have $x \equiv 8 \bmod{15}$.

\newpar{15} If $z = 0$, the equality is trivial.  Otherwise we have:

\begin{eqnarray*}
  z(x \bmod y) &=& z \left( x - y \lfloor x/y\rfloor \right) \\
  &=& zx - yz \left\lfloor\frac{zx}{zy}\right\rfloor \\
  &=& (zx) \bmod (zy).
\end{eqnarray*}

\newpar{16}  We have

\begin{eqnarray*}
  x \bmod y &=& x - y \lfloor x/y \rfloor \\
  &=& x - y \lfloor (x - z)/y + z/y\rfloor \\
  &=& x - y \left(\frac{x-z}{y} + \lfloor z/y\rfloor
  \right),\ \mbox{since $(x - z)/y$ is an integer} \\
  &=& x - y\,\frac{x-z}{y},\ \mbox{since $0 \le z < y$} \\
  &=& z
\end{eqnarray*}

\newpar{17} Let's show that $a \equiv b \bmod m$ if and only if $m$
divides $a - b$.

If $a \equiv b \bmod m$ then

\begin{eqnarray*}
  a \bmod m &=& b \bmod m \\
  a - m \lfloor a/m\rfloor &=& b - m \lfloor b/m\rfloor \\
  a - b &=& m(\lfloor a/m\rfloor - \lfloor b/m\rfloor)
\end{eqnarray*}

Reciprocally, let $q$ be a positive integer such that $a = b + q m$.
Thus,

\[ \lfloor a/m\rfloor = \lfloor b/m + q\rfloor = q + \lfloor
b/m\rfloor.\]

We then deduce that,

\[  a \bmod m = b + qm - m (q + \lfloor b/m\rfloor) = b \bmod m.\]

\medskip
So if we have $a \equiv b \bmod m$ and $x \equiv y \bmod m$, it means
that $m$ divides $a - b$ and $x - y$.  Thus, $m$ divides $(a \pm x) -
(b \pm y)$.  Which means $a \pm x \equiv b \pm y \bmod m$.

If we note $a = b + pm$ and $x = y + qm$, then we have

\[ a x = b y + m (bq + yp + mpq).\]

Thus, $m$ divides $ax - by$ so $ax \equiv by \bmod m$.

\medskip
And finally if $a \equiv b \pmod{rs}$, then $rs$ divides $a-b$.  We
then deduce that $r$ divides $a-b$ and $s$ divides $a-b$.  Thus $a
\equiv b \pmod{r}$ and $a \equiv b \pmod{s}$.

\newpar{18} Since we have $r\,\bot\, s = 1$ and

\[ s\times \frac{a - b}{s} \equiv s\times 0 \pmod{r}.\]

Using \textbf{Law B}, we deduce that $(a-b)/s \equiv 0 \pmod{r}$.
Which means that $a - b \equiv 0 \pmod{rs}$.

\newpar{19} From the extension of Euclid's algorithm, if $d$ is the
greatest common divisor between $n$ and $m$, there exist two integers
$n'$ and $m'$ such that $nn' + mm' = d$.  Since $d = 1$, we then
deduce that $n n' \equiv 1 \pmod{m}$.

\newpar{20} Suppose $ax \equiv by$, $a\equiv b$, and $a\,\bot\,m$.
From the law of inverses, there exists an integer $a'$ such that
$aa' \equiv 1 \pmod{m}$.  From \textbf{Law A}, we then deduce that $ba'
\equiv aa' \equiv 1$, thus $a'ax \equiv a'b y \pmod{m}$ is equivalent
to $x \equiv y \pmod{m}$.

\newpar{21}  From exercise 1.2.1-5, we know that each number $n > 1$
could be written as product of one or more prime numbers.  Let's show
by induction on $n$ that this representation is unique.

\medskip If $n=2$, it's trivial.  Now suppose for each integer $k$
between $2$ and $n$, the representation is unique.  Suppose that we
have two representations of $n+1$ as the product of prime numbers.
Note $p$ and $q$ respectively the greatest prime numbers in the two
representations and $a$ and $b$ the number of time they appear in the
representations.  Thus we have

\[ n = p^a \frac{n}{p^a} = q^b \frac{n}{q^b}.\]

We have $q^b (n/q^b) \equiv 0 \pmod{p}$, and $p\,\bot\,q^b$.  Thus
using \textbf{Law B}, we have $n/q^b \equiv 0 \pmod{p}$.  We then
deduce that $p$ also appears in the second representation of $n$.  But
since $q$ is the greatest prime in that representation, we deduce that
$p\le q$.  By symmetry we also have $q\le p$, thus $p=q$.

Now suppose $a \not= b$.  We treat only the case where $a < b$ since
the case $b > a$ is obtained by swapping $p$ with $q$ and $a$ with
$b$.  We then deduce that

\[ \frac{n}{p^a} = p^{b-a} \frac{n}{p^b}.\]

Since $b-a>0$, we then deduce that $p$ divides $n/p^a$ which is absurd
by the definition of $a$.

Finally, we then deduce that $a=b$.  And since the representation of
$n/p^a$ is unique by induction, we then deduce the same for $n+1$.

\newpar{22} We have $2\times 2 \equiv 0 \pmod{4}$ but, $2\not\equiv 0
\pmod{4}$.

\newpar{23} We have $4 \equiv 0 \pmod{4}$ and $4 \equiv 0 \pmod{2}$
but $4 \not\equiv 0 \pmod{8}$.

\newpar{24} Let's show that \textbf{Law A} is still true for addition
and substraction.  If $\epsilon^2 = 1$, then we have

\begin{eqnarray*}
  \left\lfloor \frac{a+\epsilon x}{m} \right\rfloor &=&
  \left\lfloor \frac{b+\epsilon y}{m} + \left(\left\lfloor
  \frac{a}{m}\right\rfloor - \left\lfloor
  \frac{b}{m}\right\rfloor \right) + \epsilon \left( \left\lfloor
  \frac{x}{m}\right\rfloor - \left\lfloor
  \frac{y}{m}\right\rfloor \right) \right\rfloor \\
  &=&   \left\lfloor \frac{b+\epsilon y}{m} \right\rfloor + \left(\left\lfloor
  \frac{a}{m}\right\rfloor - \left\lfloor
  \frac{b}{m}\right\rfloor \right) + \epsilon \left( \left\lfloor
  \frac{x}{m}\right\rfloor - \left\lfloor
  \frac{y}{m}\right\rfloor \right) \\
  &=&  \left\lfloor \frac{b+\epsilon y}{m} \right\rfloor + \frac{1}{m}
  (a - b + \epsilon(x - y))
\end{eqnarray*}

So finally,
\[ (a+\epsilon x) \bmod{m} = (b+\epsilon y) \bmod{m}.\]

The demonstration in \textbf{15.} doesn't use any integers property so
we deduce that \textbf{Law C} is always true.

\newpar{25} Use Fermat's theorem and the law of inverses.

\newpar{26} Suppose that $p$ doesn't divide $a$.  From \textbf{26}, we
have
\[ (b - 1)(b+1) = b^2 - 1 \equiv 0 \bmod{p}\]

Using \textbf{Law B}, we then deduce that $b \equiv 1 \bmod{p}$ or $b
\equiv -1 \bmod{p}$.

\newpar{27} Since $p$ is prime, then $p\,\bot\,i = 1$ for $i$ varying
between $1$ and $p-1$.  Thus $\varphi(p) = p-1$.

Suppose $e$ is a positive integer.  The integers between $0$ and
$p^e-1$ that share a common divisor with $p^e$ are:
\[ 0, p, 2p, \ldots, \left\lfloor \frac{p^e-1}{p}\right\rfloor p =
(p^{e-1}-1)p.\]

We then deduce that $\varphi(p^e) = p^e - p^{e-1}$.

\newpar{28} Note $I(n)$ the integers between $0$ and $n-1$ which are
inversable.  That is $a \in I(n)$ then there exists $a'$ such that
$aa' \equiv 1 \bmod{n}$.  Note that $a'$ is also an element of $I(n)$.
Plus if $a\,\bot\,n = 1$, then the application $x \mapsto ax$ is a
permutation of $I(n)$.  We then have

\[ \prod_{x \in I(n)}(ax) \equiv \prod_{x \in I(n)}x \equiv a^{\varphi(n)}
\prod_{x\in I(n)} x.\]

We then deduce that $a^{\varphi(n)} \equiv 1 \pmod{n}$ since $I(n)$ is
stable under multiplication.

\newpar{29}  \subpar{a} Note that since $r\,\bot\,s = 1$ then
$rs\not=0$. Thus $f$ is defined and is already multiplicative
regardless of $r$ and $s$.

\subpar{b} Let $k$ be a positive integer such that $k^2$ divides
$rs$.  Since $r\,\bot\,s = 1$, then $k^2$ divides $r$ or (exclusive)
$k^2$ divides $s$.  We then deduce easily

\begin{eqnarray*}
  f(rs) &=& [rs\ \mbox{is not divisible by $k^2$ for any $k>0$}] \\
  &=& [\mbox{$r$ is not divisible by $k^2$ and $s$ is not divisible by
      $k^2$}] \\
  &=& f(r)f(s)
\end{eqnarray*}

\subpar{c} $r\,\bot\,s = 1$ and $n$ has a unique representation as a
product of primes.

\subpar{d} If $f$ and $g$ are two multiplicative functions, we have
\[ f(rs)g(rs) = (f(r)g(r))(f(s)g(s)).\]

\newpar{30}  Suppose that $r\,\bot\,s = 1$.  If $n$ is a positive
integer, note $\mathbf{Z/nZ}^*$ the elements of $\mathbf{Z/nZ}$ that
are invertible.  Note $f$ the application
\begin{eqnarray*}
  f: \mathbf{Z/rsZ}^* &\longrightarrow& \mathbf{Z/rZ}^* \times
  \mathbf{Z/sZ}^* \\
  x \bmod{rs} &\longmapsto& (x \bmod{r}, x\bmod{s})
\end{eqnarray*}

Let's show that $f$ is a bijection.  Let $x$ and $x'$ be two integers
relatively primes to $rs$ and such that $f(x\bmod{rs}) =
f(x'\bmod{rs})$.  We then deduce that $r$ and $s$ divides $x - x'$.
We then deduce from \textbf{Law D} that $rs$ divides $x-x'$.  Thus $f$
is injective.

Let $(y\bmod r, z\bmod s)$ be an element of $\mathbf{Z/rZ}^*\times
\mathbf{Z/sZ}^*$.  From \textbf{19.}, there exist two integers $n$ and
$m$ such that $n r + m s = 1$.  We then deduce that
\[y-z = nr(y-z) + ms(y-z).\]

If we note $x = y + nr(z-y) = z + ms(y-z)$.  Then we have
\[f(x\bmod{rs}) = (y \bmod r, z\bmod s).\]
Plus since $\gcd(x, r) = \gcd(y, r) = 1$ and $\gcd(x, s) = \gcd(z, s)
= 1$, we then deduce that $x\,\bot\,rs$.  Hence, $f$ is a surjection.

Thus $f$ is a bijection from $\mathbf{Z/rsZ}^*$ to
$\mathbf{Z/rZ}^*\times\mathbf{Z/sZ}^*$, thus \[\varphi(rs) = \varphi(r)
\varphi(s).\]

Thus we have using \textbf{27.}
\[ \varphi(10^6) = \varphi(2^6) \varphi(5^6) = (2^6 - 2^5)(5^6 - 5^5)
= 400000\]

We use the following method to compute $\varphi(n)$.  If $n =
p_1^{n_1}\cdots p_k^{n_k}$ where the integers $p_1, \ldots, p_k$ are
prime numbers, we have
\[ \varphi(n) = \prod_{i=1}^kp_i^{n_i-1}(p_i-1).\]

\newpar{31} Suppose that $f(n)$ is multiplicative.  If $r\,\bot\,s$,
then we have
\begin{eqnarray*}
  g(rs) &=& \sum_{d\backslash rs}f(d) \\
  &=& \sum_{d\backslash r}\sum_{d'\backslash s}f(dd') \\
  &=& \sum_{d\backslash r} \sum_{d'\backslash s} f(d)
  f(d'),\ \mbox{because $d\,\bot\,d'$} \\
  &=& g(r) g(s)
\end{eqnarray*}

\newpar{32} We have

\begin{eqnarray*}
  \sum_{d \char92 n} \sum_{c \char92 d} f(c, d) &=& \sum_{d \char92
    n\,\mathrm{and}\,c\char92 d} f(c, d) \\
  &=& \sum_{(cd')\char92 n}f(c, cd'),\,\mbox{if $d = cd'$} \\
  &=& \sum_{c \char92 n} \sum_{d'\char92 (n/c)} f(c, cd')
\end{eqnarray*}

\newpar{33} \subpar{a}  If $m$ and $n$ have the same parity, $n
\pm m$ is even thus
\[\left\lfloor \frac{1}{2}(n+m)\right\rfloor + \left\lfloor
\frac{1}{2}(n-m+1)\right\rfloor = \frac{n+m}{2} + \frac{n-m}{2} = n.\]

Otherwise $n\pm m$ is odd and
\[\left\lfloor \frac{1}{2}(n+m)\right\rfloor + \left\lfloor
\frac{1}{2}(n-m+1)\right\rfloor = \frac{n+m-1}{2} + \frac{n-m+1}{2} =
n.\]

\subpar{b} If $m$ and $n$ have the same parity
\[\left\lceil \frac{1}{2}(n+m)\right\rceil + \left\lceil
\frac{1}{2}(n-m+1) \right\rceil = \frac{n+m}{2} + \frac{n-m}{2} + 1 =
n+1\]

Otherwise
\[\left\lceil \frac{1}{2}(n+m)\right\rceil + \left\lceil
\frac{1}{2}(n-m+1) \right\rceil = \frac{n+m-1}{2} + 1 +
\frac{n-m+1}{2} = n+1.\]

\newpar{34} Let $n$ be an integer such that
\[ n \le \log_b \lfloor x\rfloor < n+1\]

Since $b>1$, we have

\[ b^n \le \lfloor x\rfloor < b^{n+1}.\]

Given that the all the numbers in the inequalities are integers, the
preceding inequalities are equivalent to
\[ b^n \le \lfloor x\rfloor +1\le b^{n+1}.\]

Thus,
\[ b^n \le \lfloor x\rfloor \le x < \lfloor x\rfloor + 1 \le b^{n+1}\]

$x = b$ implies that $n=1$ and $\lfloor b\rfloor = b$.
Thus $b$ is in an integer greater than $1$.

And taking the logarithm

\[ n \le \log_b x < n+1.\]

We then deduce that $n = \lfloor \log_b \lfloor x\rfloor \rfloor =
\lfloor \log_b x\rfloor$ if and only if $b$ is an integer greater than
$1$.

\newpar{35} First let's treat the special case where $m=0$.  We have
\[ n \left\lfloor \frac{x}{n}\right\rfloor \le x < \lfloor x\rfloor +
1\]

Since the leftmost and rightmost members in the previous inequalities
are integers, we then deduce that

\[ n \left\lfloor \frac{x}{n}\right\rfloor \le \lfloor x\rfloor.\]
Hence \[ \left\lfloor \frac{x}{n}\right\rfloor \le \left\lfloor
\frac{\lfloor x\rfloor}{n}\right\rfloor\]

Since the reverse of this inequality is always true, we then deduce
that
\[ \left\lfloor \frac{x}{n}\right\rfloor = \left\lfloor
\frac{\lfloor x\rfloor}{n}\right\rfloor\]

Thus we have
\[ \left\lfloor \frac{x+m}{n}\right\rfloor =
\left\lfloor \frac{\lfloor x+m\rfloor}{n}\right\rfloor =
\left\lfloor \frac{\lfloor x\rfloor + m}{n}\right\rfloor\]

\newpar{36} We have

\begin{eqnarray*}
  \sum_{k=1}^{2n} \lfloor k/2\rfloor &=& \sum_{l=0}^{n-1} (\lfloor (2l)/2
  \rfloor + \lfloor (2l+1)/2\rfloor) + n \\
  &=& 2 \sum_{l=0}^{n-1} l + n \\
  &=& (n-1)n + n \\
  &=& n^2 \\
  &=& \lfloor (2n^2)/4\rfloor
\end{eqnarray*}

And we then deduce

\begin{eqnarray*}
  \sum_{k=1}^{2n+1}\lfloor k/2\rfloor &=& \sum_{k=1}^{2n}\lfloor
  k/2\rfloor + \lfloor (2n+1)/2\rfloor \\
  &=& n^2 + n \\
  &=& \lfloor n^2 + n + 1/4\rfloor \\
  &=& \lfloor (n+1/2)^2\rfloor \\
  &=& \lfloor (2n+1)^2/4\rfloor
\end{eqnarray*}

Hence,
\[ \sum_{k=1}^n \lfloor k/2\rfloor = \lfloor n^2/4\rfloor.\]

We have also

\begin{eqnarray*}
  \sum_{k=1}^n \lceil k/2\rceil &=& \sum_{0\le k\le n/2} \lceil
  (2k)/2\rceil + \sum_{0\le k\le (n-1)/2} \lceil (2k+1)/2\rceil \\
  &=& \sum_{0\le k\le n/2}\lfloor (2k)/2\rfloor +
  \sum_{0\le k\le (n-1)/2}(\lfloor (2k+1)/2\rfloor + 1) \\
  &=& \sum_{k=1}^n \lfloor k/2\rfloor + \lfloor (n+1)/2\rfloor \\
  &=& \lfloor n^2/4\rfloor + \lfloor (n+1)/2\rfloor \\
  &=& \lfloor (n^2 + 2n + 2)/4\rfloor \\
  &=& \left\lfloor \frac{(n+1)^2+1}{4}\right\rfloor
\end{eqnarray*}

\newpar{37} We have

\begin{eqnarray*}
  \sum_{0\le k<n}\left\lfloor \frac{mk + x}{n}\right\rfloor &=&
  \sum_{0\le k<n}\frac{mk+x}{n} - \sum_{0\le k<n}\left\{
  \frac{mk+x}{n}\right\} \\ &=&
  x + \frac{m(n-1)}{2} - \sum_{0\le k<n}\left\{\frac{mk+x}{n}\right\}
\end{eqnarray*}

Note $d$ the greatest common divisor of $m$ and $n$ and note $m' =
m/d$ and $n' = n/d$.  We have

\begin{eqnarray*}
  \sum_{0\le k<n}\left\{ \frac{mk+x}{n} \right\} &=& \sum_{0\le k<n}
  \left\{ \frac{m'k}{n'} + \frac{x}{n} \right\} \\
  &=& \sum_{l=0}^{d-1} \sum_{k=ln'}^{(l+1)n'-1}\left\{ \frac{m'k}{n'}
  + \frac{x}{n} \right\} \\
  &=& \sum_{l=0}^{d-1} \sum_{k=0}^{n'-1} \left\{ \frac{m'k}{n'} + lm'
  + \frac{x}{n} \right\} \\
  &=& \sum_{l=0}^{d-1}\sum_{k=0}^{n'-1} \left\{ \frac{m'k}{n'} +
  \frac{x}{n} \right\} \\
  &=& d \sum_{k=0}^{n'-1}\left\{ \frac{m'k}{n'} + \frac{x}{n} \right\}
\end{eqnarray*}

since $m'\,\bot\,n' = 1$, then $k \mapsto m'k$ is a permutation of
$Z/n'Z$.  Hence,

\begin{eqnarray*}
  \sum_{0\le k<n}\left\{ \frac{mk+x}{n}\right\} &=&
  d \sum_{0\le k <n'}\left\{ \frac{k}{n'} + \frac{x}{n}\right\} \\ &=&
  d \sum_{0\le k<n'} \left\{ \frac{k}{n'} + \frac{x/d}{n'}\right\}
  \\ &=& d \sum_{0\le k<n'}\left\{ \frac{k + \lfloor x/d\rfloor}{n'} +
  \frac{\{ x/d \}}{n'}\right\} \\ &=&
  d \sum_{0\le k<n'}\left\{ \frac{k}{n'} + \frac{\{x/d\}}{n'} \right\}
  \\ &=& d \sum_{0\le k<n'} \left( \frac{k + \{x/d\}}{n'} \right)
\end{eqnarray*}

since $k \mapsto k + \lfloor x/d\rfloor$ is a permutation of $Z/n'Z$
and $0 \le k + \{x/d\} < n'$ for $0\le k \le n'-1$.  Thus finally

\begin{eqnarray*}
  \sum_{0\le k<n}\left\{ \frac{mk+x}{n} \right\} &=&
  d \left( \frac{n'-1}{2} + \{x/d\} \right) \\ &=&
  \frac{d(n'-1)}{2} + x - d \lfloor x/d\rfloor
\end{eqnarray*}

At last, we deduce that

\begin{eqnarray*}
  \sum_{0\le k<n}\left\lfloor \frac{mk+x}{n} \right\rfloor &=& x +
  \frac{m(n-1)}{2} - \frac{d(n'-1)}{2} - x + d \lfloor x/d\rfloor
  \\ &=& \frac{(m-1)(n-1)}{2} + \frac{d-1}{2} + d \lfloor x/d\rfloor
\end{eqnarray*}

\newpar{38}  Note $n$ the integer that verifies

\[ x + \frac{n-1}{y} < \lfloor x\rfloor + 1 \le x + \frac{n}{y} .\]

Thus we have $n = \lceil \lfloor x + 1\rfloor y - x y\rceil$.   On the
other hand, we have

\begin{eqnarray*}
  \sum_{0\le k <y}\left\lfloor x + \frac{k}{y}\right\rfloor &=&
  \sum_{0\le k\le\lceil y\rceil -1}\left\lfloor x +
  \frac{k}{y}\right\rfloor \\ &=&
  \sum_{0\le k\le n-1}\left\lfloor x + \frac{k}{y}\right\rfloor +
  \sum_{n\le k\le \lceil y\rceil - 1}\left\lfloor x +
  \frac{k}{y}\right\rfloor \\ &=&
  n\lfloor x\rfloor + \lfloor x+1\rfloor(\lceil y\rceil - n) \\ &=&
  \lfloor x+1\rfloor \lceil y\rceil - n \\ &=&
  \lfloor x+1\rfloor \lceil y\rceil + \lfloor xy - \lfloor x+1\rfloor
  y\rfloor \\ &=&
  \lfloor xy + \lfloor x+1\rfloor(\lceil y\rceil - y)\rfloor
\end{eqnarray*}

\newpar{39} \subpar{a} We have
\begin{eqnarray*}
  \sum_{k=0}^{n-1} \left(x + \frac{k}{n} - \frac{1}{2} \right) &=&
  \left(x - \frac{1}{2}\right) n + \frac{n-1}{2} \\
  &=& nx - \frac{1}{2}
\end{eqnarray*}

\subpar{b} Only one of the numbers $x + \frac{k}{n}$ for $0\le k\le
n-1$ could take an integer value since they're contained in an
interval of length less than $1$.  And if one of them is an integer,
we deduce easily that $nx$ is also an integer.

Suppose that $n x$ is an integer equal to $m$.  Thus
\[ x = \left\lfloor \frac{m}{n}\right\rfloor + \frac{m \bmod n}{n}\]

If $m \bmod n = 0$ then $x$ is an integer.  Otherwise, $x +
\frac{k}{n}$ is an integer where $0 < k = n - m \bmod n < n$.

Finally, we then deduce that $nx$ is an integer if and only if exactly
one of the numbers $x + \frac{k}{n}$ for $0\le k\le n-1$ is an
integer.  Thus

\[ \sum_{0\le k\le n-1} \left[x+\frac{k}{n}\ \mbox{is an
    integer}\right] = [ nx\ \mbox{is an integer}]\]

\subpar{c} If $x \le 0$ then $x + \frac{k}{n} < 1$ for $0\le k\le
n-1$.  Thus, none of the numbers $x + \frac{k}{n}$ is a positive
integer.  We then deduce that if one of these numbers is a positive
integer, then $x > 0$.  And thus $nx$ is also a positive integer.

Conversely, if $nx$ is a positive integer, then $x$ is positive
integer and we conclude that exactly one of the numbers
$x+\frac{k}{n}$ is a positive integer using the same reasoning as in
(b).  Thus

\[ \sum_{0\le k\le n-1} \left[x+\frac{k}{n}\ \mbox{is a positive
    integer}\right] = [ nx\ \mbox{is a positive integer}]\]

\subpar{d} Let $m$ and $m'$ be two integers, $r$ and $r'$ two
rationals and $k$ and $k'$ two other integers between $0$ and $n-1$
such that
\begin{eqnarray*}
  x + \frac{k}{n} &=& r\pi + m \\
  x + \frac{k'}{n} &=& r'\pi + m'
\end{eqnarray*}

By subtracting  the two equations we obtain
\[ \frac{k-k'}{n} = (r-r')\pi + m-m'.\]

Since $\pi$ is irrational, we deduce that $r=r'$.  And given that
$|k-k'| < n$, we then deduce that $|m - m'| < 1$.  Since that last
expression is an integer, we deduce that $m = m'$ and finally $k=k'$.

We then deduce that only one of the numbers $x + \frac{k}{n}$ for
$0\le k\le n-1$ could take a value of the form $r \pi + m$ where $r$
is rational and $m$ is an integer.  And in that case $n x = rn \pi +
nm-k$ has also the same form.

Conversely, suppose that there's a rational $r$ and an integer $m$
such that $nx = r\pi + m$.  Then,
\[ x = \frac{r}{n} \pi + \left\lfloor \frac{m}{n}\right\rfloor +
\frac{m\bmod n}{n}\]

If $m\bmod n = 0$ then $x$ has the desired form.  Otherwise
\[ x + \frac{n-m \bmod n}{n} = \frac{r}{n} \pi + \left\lfloor
\frac{m}{n}\right\rfloor + 1,\ \mbox{where $1\le n - m \bmod n<n$}\]

Finally, we deduce that $nx$ is of the form $r\pi + m$ if and only if
exactly one of the numbers $x + \frac{k}{n}$ for $0\le k \le n-1$ is
of the same form.

\subpar{e}  Just note that $rn > 0$ if and only if $r > 0$.  Thus the
reasoning in (c) holds true if we restrict $r$ to be positive.

If $m>0$, then $nm-k>0$ for $0\le k<n$.  Moreover if $m \bmod n = 0$,
then $\lfloor m/n\rfloor > 0$.  Thus the reasoning in (c) remains true
if we restrict $m$ to be positive.

\subpar{f} We have
\begin{eqnarray*}
  \sin \pi n x &=& \frac{e^{i\pi nx} - e^{-i\pi nx}}{2i} \\
  &=& e^{-i\pi nx} \frac{e^{i2\pi nx} - 1}{2i} \\
  &=& \frac{e^{-i\pi nx}}{2i} \prod_{k=0}^{n-1}(e^{i2\pi x} -
  e^{i2k\pi/n}) \\
  &=& \frac{e^{-i\pi nx}}{2i}
  \prod_{k=0}^{n-1}e^{i\pi(x+k/n)}(e^{i2\pi (x-k/n)} -
  e^{i2k\pi(x-k/n)}) \\
  &=& \frac{e^{i\pi(n-1)/2}}{2i} \prod_{k=0}^{n-1} 2i \sin\pi\left(x
  - \frac{k}{n}\right) \\
  &=& \frac{i^{n-1}}{2i} (2i)^n \prod_{k=0}^{n-1} \sin\pi\left(x
  - \frac{k}{n}\right) \\
  \sin \pi n x &=& 2^{n-1} \prod_{k=0}^{n-1} \sin\pi\left(x
  - \frac{k}{n}\right)
\end{eqnarray*}

So finally, we have

\begin{eqnarray*}
  \sum_{k=0}^{n-1} \log\left|2 \sin\pi \left(x +
  \frac{k}{n}\right)\right| &=& \log \left| 2^n \prod_{k=0}^{n-1}
  \sin\pi \left(x + \frac{k}{n}\right)\right| \\
  &=& \log \left| 2^n \prod_{k=0}^{n-1} \sin\pi\left(-x -
  \frac{k}{n}\right)\right| \\
  &=& \log| 2 \sin\pi n(-x)| \\
  &=& \log| 2 \sin\pi n x|
\end{eqnarray*}

\subpar{g} and \subpar{h} could be proved using the linearity of
summation.

\subpar{i}  Let $m$ be the integer such that
\[ \{x\} + \frac{m-1}{n} < 1 \le \{x\} + \frac{m}{n}.\]
That is, $m = \lceil n(1-\{x\})\rceil$.  We then have

\begin{eqnarray*}
  \sum_{k=0}^{n-1}g\left(x+\frac{k}{n}\right) &=&
  \sum_{k=0}^{n-1}f\left(\left\{x+\frac{k}{n}\right\}\right) \\
  &=&   \sum_{k=0}^{m-1}f\left(\left\{x+\frac{k}{n}\right\}\right) +
  \sum_{k=m}^{n-1}f\left(\left\{x+\frac{k}{n}\right\}\right) \\
  &=& \sum_{k=0}^{m-1}f\left(\{x\} + \frac{k}{n}\right) +
  \sum_{k=m}^{n-1}f\left(\{x\}+\frac{k}{n}-1\right) \\
  &=& \sum_{k=n-m}^{n-1}f\left(\{x\}+\frac{m-n+k}{n}\right) +\\
  &&\sum_{k=0}^{n-m-1}f\left(\{x\} + \frac{m-n+k}{n}\right) \\
  &=& \sum_{k=0}^{n-1}f\left(\{x\}+\frac{m-n}{n} + \frac{k}{n}\right)
  \\
  &=& f\left(n\left(\{x\} + \frac{m -
    n}{n}\right)\right),\ \mbox{since $f$ is replicative} \\
  &=& f(n\{x\} + \lceil n(1-\{x\})\rceil - n ) \\
  &=& f(n\{x\} + \lceil - n\{x\}\rceil) \\
  &=& f(n\{x\} - \lfloor n\{x\}\rfloor) \\
  &=& f(\{n\{x\}\}) \\
  &=& f(\{nx - n\lfloor x\rfloor\}) \\
  &=& f(\{nx\})
\end{eqnarray*}

We then deduce that $g$ is replicative.

\newpar{41} Let $n$ be a positive integer and $k$ the integer
verifying
\[ \sum_{i=1}^k i \le n < \sum_{i=1}^{k+1}i.\]
We then have $a_n = k$.  Rewriting the preceding inequalities, we have
\[ \frac{k(k+1)}{2} \le n < \frac{(k+1)(k+2)}{2}.\]

Resolving the equation $ \frac{x(x+1)}{2} = n $, we find a positive
solution for \[x = \frac{\sqrt{1+8n}-1}{2}\].

Given that $x \mapsto \frac{x(x+1)}{2}$ is an increasing function for
$x\ge 0$, we then deduce that
\[ a_n = k = \left\lfloor \frac{\sqrt{1+8n}-1}{2}\right\rfloor.\]

\newpar{42} \subpar{a} We have
\begin{eqnarray*}
  \sum_{k=1}^n a_k &=& \sum_{k=1}^n (ka_k - (k-1)a_k) \\
  &=& \sum_{k=1}^n ka_k - \sum_{k=0}^{n-1}ka_{k+1} \\
  &=& na_n - \sum_{k=1}^{n-1} k(a_{k+1} - a_k)
\end{eqnarray*}

\subpar{b}  We have
\begin{eqnarray*}
  \sum_{k=1}^n\lfloor \log_b k\rfloor &=& \sum_{l=0}^{\lfloor \log_b
    n\rfloor -1} \sum_{b^l \le k < b^{l+1}} \lfloor \log_b k\rfloor +
  \sum_{k = b^{\lfloor \log_b n\rfloor}}^n \lfloor \log_b k\rfloor \\
  &=& \sum_{l=0}^{\lfloor \log_b n\rfloor -1} l(b^{l+1}-b^l) +
  \lfloor \log_b n\rfloor (n - b^{\lfloor \log_b n\rfloor}+1) \\
  &=& - \sum_{l=1}^{\lfloor \log_b n\rfloor} b^l + (n+1) \lfloor
  \log_b n\rfloor,\ \mbox{from (a)}\\
  &=& (n+1)\lfloor \log_b n\rfloor - (b^{\lfloor \log_b n\rfloor +1}-b)(b-1)
\end{eqnarray*}

\newpar{43} We have
\begin{eqnarray*}
  \sum_{k=1}^n \lfloor \sqrt{k}\rfloor &=&
  \sum_{l=0}^{\lfloor \sqrt{n}\rfloor -1} \sum_{l^2 \le k < (l+1)^2}
  \lfloor \sqrt{k}\rfloor + \sum_{k=\lfloor \sqrt{n}\rfloor^2}^n
  \lfloor \sqrt{k}\rfloor \\
  &=& \sum_{l=0}^{\lfloor \sqrt{n}\rfloor -1} l((l+1)^2 - l^2) +
  \lfloor \sqrt{n}\rfloor(n - \lfloor \sqrt{n}\rfloor^2+1) \\
  &=& - \sum_{l=1}^{\lfloor\sqrt{n}\rfloor} l^2 + (n+1)\lfloor
  \sqrt{n}\rfloor \\
  &=& - \frac{\lfloor \sqrt{n}\rfloor (\lfloor \sqrt{n}\rfloor + 1)
    (2\lfloor\sqrt{n}\rfloor+1)}{6} + (n+1)\lfloor \sqrt{n}\rfloor \\
  &=& \lfloor\sqrt{n}\rfloor \left(n+1 - \frac{(\lfloor\sqrt{n}\rfloor
    + 1)(2\lfloor\sqrt{n}\rfloor+1)}{6}\right)
\end{eqnarray*}


\newpar{44}  Using \textbf{38} with $x = n/b^{k+1}$ and $y = b$, we have
\[ \sum_{1\le j<b} \left\lfloor \frac{n}{b^{k+1}} +
\frac{j}{b}\right\rfloor = \left\lfloor\frac{n}{b^k}\right\rfloor -
\left\lfloor\frac{n}{b^{k+1}}\right\rfloor.\]

Hence
\[ \sum_{k\ge 0} \sum_{1\le j<b} \left\lfloor \frac{n}{b^{k+1}} +
\frac{j}{b}\right\rfloor = n.\]

\newpar{45} We have
\begin{eqnarray*}
  \sum_{0\le j<n}f\left( \left\lfloor \frac{mj}{n} \right\rfloor
  \right) &=& \sum_{r=0}^{m-1} \sum_{\frac{nr}{m} \le j <
    \frac{n(r+1)}{m}} f\left( \left\lfloor \frac{mj}{n} \right\rfloor
  \right) \\
  &=& \sum_{r=0}^{m-1} \sum_{\left\lceil \frac{nr}{m}\right\rceil \le
    j < \left\lceil \frac{n(r+1)}{m}\right\rceil} f(r) \\
  &=& \sum_{r=0}^{m-1} f(r) \left( \left\lfloor \frac{n(r+1)}{m}
  \right\rfloor - \left\lfloor \frac{nr}{m} \right\rfloor \right) \\
  &=& \sum_{1\le r <m+1} f(r-1) \left\lfloor \frac{nr}{m}
  \right\rfloor - \sum_{0\le r < m} f(r)\left\lfloor \frac{nr}{m}
  \right\rfloor \\
  &=& \sum_{0\le r < m} (f(r-1) - f(r)) \left\lceil \frac{rn}{m}
  \right\rceil + n f(m-1)
\end{eqnarray*}

Thus taking $f(x) = C^k_{x+1}$, we have
\[  \sum_{0\le j<n} C^k_{\lfloor mj/n\rfloor + 1} =
  \sum_{0\le r<m} \left\lceil \frac{rn}{m} \right\rceil
  \left( C^k_r - C^k_{r+1}\right) + n C^k_m \]

But on the other hand
\begin{eqnarray*}
  C^k_r - C^k_{r+1} &=& \frac{r!}{k!(r-k)!} -
  \frac{(r+1)!}{k!(r+1-k)!} \\
  &=& \frac{r!}{k!(r-k)!} \left(1 - \frac{r+1}{r+1-k}\right) \\
  &=& - C^{k-1}_r
\end{eqnarray*}

Hence
\[\sum_{0\le j<n}C^k_{\lceil mj/n\rceil + 1} + \sum_{0\le j <m}
C^{k-1}_j = n C^k_m\]

\newpar{46} We have
\begin{eqnarray*}
  \sum_{0\le j< \alpha n} f\left( \left\lfloor \frac{mj}{n}
  \right\rfloor \right) &=&
  \sum_{0\le r < \lceil \alpha m\rceil-1}\ \sum_{ nr/m\le j < n(r+1)/m} f(\lfloor
  mj/n\rfloor) + \\
  && \sum_{n(\lceil \alpha m\rceil - 1)/m \le j < \alpha n} f(\lfloor mj/n\rfloor) \\
  &=& \sum_{r=0}^{\lceil \alpha m\rceil - 2}
  \sum_{\lceil nr/m \rceil \le j < \lceil n(r+1)/m\rceil} f(r) + \\
  && \sum_{\lceil n(\lceil \alpha\rceil - 1)/m \rceil \le j < \lceil \alpha n\rceil}
  f(\lceil \alpha m \rceil - 1)\\
  &=& \sum_{r=0}^{\lceil \alpha m\rceil -2}
  f(r)(\lceil n(r+1)/m\rceil - \lceil nr/m\rceil) + \\
  && f(\lceil \alpha m\rceil - 1)(\lceil \alpha n\rceil -
  \lceil n(\lceil\alpha\rceil - 1)/m\rceil) \\
  &=& \sum_{r=1}^{\lceil \alpha m\rceil - 1} \lceil nr/m\rceil (f(r-1)
  - f(r)) + f(\lceil \alpha m\rceil - 1) \lceil \alpha n\rceil
\end{eqnarray*}

Thus finally
\[ \sum_{0\le j<\alpha n}f\left( \left\lfloor \frac{mj}{n}
\right\rfloor \right) = \sum_{0\le r < \alpha m}\left\lceil
\frac{nr}{m} \right\rceil(f(r-1)-f(r)) + f(\lceil \alpha m\rceil -
1)\lceil \alpha n\rceil\]

\newpar{47} \subpar{a} Let $k$ and $k'$ be two integers such that
$0<k,k'< p/2$ and
\[ (-1)^{\lfloor 2kq/p\rfloor}(2kq \bmod p) = (-1)^{\lfloor
  2k'q/p\rfloor}(2k'q \bmod p).\]

Given that $q \bot p$, we then deduce that $k' \equiv \pm k \bmod p$,
which is equivalent to $k' = k'\bmod p = \pm k \bmod p$ since
$0<k'<p/2$.  But we have
\begin{eqnarray*}
  -k \bmod p &=& -k - p \left\lfloor \frac{-k}{p}\right\rfloor \\
  &=& (p - k) - p \left\lfloor \frac{p-k}{p}\right\rfloor \\
  &=& p-k,\ \mbox{since $p/2 < p-k < p$} \\
  &>& \frac{p}{2} \\
  &>& k'
\end{eqnarray*}

We then deduce that $k' = k\bmod p = k$.  Thus if we note
\[ f(k) = (-1)^{\lfloor 2kq/p\rfloor}(2kq\bmod p)\]
then $f$ is injective for $0 < k < p/2$.  Moreover, given that
\[ 2kq = p\lfloor 2kq/p\rfloor + 2kq \bmod p.\]

and $p$ is an odd prime number, we deduce that $\lfloor 2kq/p\rfloor$
and $2kq \bmod p$ have the same parity.  If they're both even then
$f(k) \equiv (2kq \bmod p) \bmod p$.  Otherwise, $f(k) \equiv (p - 2kq
\bmod p) \bmod p$.

Since $f$ is injective, we then deduce that its image is a permutation
of $2, 4, \ldots, p-1 \pmod p$.  Hence
\begin{eqnarray*}
  \prod_{0 < k < p/2} (-1)^{\lfloor 2kq/p\rfloor} 2kq
  &\equiv& \left(\prod_{0 < k <p/2} 2k \right) \bmod p \\
  q^{\frac{p-1}{2}} &\equiv& (-1)^{\sum_{0<k<p/2} \lfloor
    2kq/p\rfloor} \bmod p
\end{eqnarray*}

\subpar{b} We have
\begin{eqnarray*}
  \sigma &=& \sum_{0 \le k<p/2}\lfloor 4k/p\rfloor \\
  &=& \sum_{p/4 \le k < p/2} \lfloor 4k/p\rfloor \\
  &=& \sum_{\lceil p/4\rceil \le k < \lceil p/2\rceil} 1 \\
  &=& \lceil p/2\rceil - \lceil p/4\rceil \\
  &=& \frac{p+1}{2} + \left\lfloor \frac{-p}{4}\right\rfloor \\
  &=& \left\lfloor \frac{p+1}{2} - \frac{p}{4}\right\rfloor \\
  &=& \left\lfloor \frac{p+2}{4}\right\rfloor
\end{eqnarray*}

Hence \[ \left(\frac{2}{q}\right) = (-1)^{\lfloor (p+2)/4\rfloor}.\]

\subpar{c} We have
\begin{eqnarray*}
  \sum_{0\le k<p/2} \lfloor 2kq/p\rfloor &=&
  \sum_{0\le k<p/4} \lfloor 2kq/p\rfloor + \sum_{p/4\le k<p/2} \lfloor
  2kq/p\rfloor \\
  &=& \sum_{0\le k<p/4} \lfloor 2kq/p\rfloor +
  \sum_{-1/2 \le k \le (p-2)/4} \left\lfloor \frac{2\left(\frac{p-1}{2} -
    k\right)q}{p}\right\rfloor \\
  &=& \sum_{0\le k<p/4} \lfloor 2kq/p\rfloor +
  \sum_{0\le k\le (p-2)/4} \left( q + \left\lfloor
  -\frac{(2k+1)q}{p}\right\rfloor\right)
\end{eqnarray*}

$(2k+1)q/p$ is not an integer since $p$ is prime, $0 < 2k+1 < p$ and
$q\,\bot\,p$, thus
\begin{eqnarray*}
  \left\lfloor -\frac{(2k+1)q}{p}\right\rfloor &=&
  - \left\lceil \frac{(2k+1)q}{p}\right\rceil =
  - \left( \left\lfloor \frac{(2k+1)q}{p}\right\rfloor + 1\right)
\end{eqnarray*}

Given that $q$ is odd, we have finally
\begin{eqnarray*}
  \sum_{0\le k<p/2}\lfloor 2kq/p\rfloor &\equiv& \left(
  \sum_{0\le k<p/4} \lfloor 2kq/p\rfloor +
  \sum_{0\le k\le (p-2)/4} \lfloor (2k+1)q/p\rfloor \right) \bmod 2 \\
  &\equiv& \sum_{0\le k <p/2} \lfloor kq/p\rfloor \pmod 2
\end{eqnarray*}

\subpar{d} From exercise \textbf{46}, we deduce
\begin{eqnarray*}
  \sum_{0\le j<p/2} \lfloor kq/p\rfloor &=&
  \sum_{0\le r<q/2} \lfloor pr/q \rfloor ((r-1) - r) +
  (\lceil q/2\rceil - 1) \lceil p/2\rceil \\
  &=& - \sum_{0\le r<q/2} \lfloor pr/q\rfloor + \frac{(q-1)(p+1)}{4}
\end{eqnarray*}

And finally
\[ \sum_{0\le j<p/2} \lfloor kq/p\rfloor + \sum_{0\le j<q/2} \lfloor
pj/q\rfloor \equiv \frac{(q-1)(p-1)}{4} \pmod 2.\]
Hence if $p$ and $q$ are distinct odd prime numbers, we have
\[ \left(\frac{p}{q}\right) \left(\frac{q}{p}\right) =
(-1)^{(p-1)(q-1)/4}.\]

\newpar{48} \subpar{a} If $n>0$, then we have
\begin{eqnarray*}
  \left\lfloor \frac{m+n-1}{n} \right\rfloor - \left\lceil
  \frac{m}{n}\right\rceil &=& \left\lfloor
  \frac{m+n-1}{n} + \left\lfloor -\frac{m}{n}\right\rfloor
  \right\rfloor \\
  &=& \left\lfloor \frac{n-1 - (-m) \bmod n}{n} \right\rfloor \\
  &=& 0
\end{eqnarray*}

But the equality doesn't generally holds when $n<0$.

\subpar{b} Using (a), we have
\begin{eqnarray*}
  \left\lfloor \frac{n+2-\lfloor n/25\rfloor}{3}\right\rfloor &=&
  \left\lceil \frac{n - \lfloor n/25\rfloor}{3}\right\rceil \\
  &=& \left\lceil \frac{n + \lceil -n/25\rceil}{3}\right\rceil \\
  &=& \left\lceil \frac{\lceil 24n/25\rceil}{3}\right\rceil \\
  &=& \lceil 8n/25\rceil \\
  &=& \left\lfloor \frac{8n+24}{25}\right\rfloor
\end{eqnarray*}

\newpar{49} If $n$ is an integer we have $f(x+n) = f(x) + n$.  Since
$f$ is integer-valued, we have
\[ f(0) = f(f(0)) = f(0 + f(0)) = f(0) + f(0)\]
Hence
\[ f(0) = 0\ \mbox{and}\ f(n) = n\ \mbox{if $n$ is an integer}.\]

If $f(1/2) \le 0$, then
\begin{eqnarray*}
  f(1/2) &=& f\left( \frac{f((1-2f(1/2))1/2)}{1-2f(1/2)} \right) \\
  &=& f\left( \frac{f(1/2) - f(1/2)}{1 - 2f(1/2)} \right) \\
  &=& 0
\end{eqnarray*}

If $n > 1$, then
\[ f\left(\frac{1}{n-1}\right) = f\left(
\frac{f(1+1/(n-1))}{n}\right) =
f\left( \frac{f(1/(n-1)) + 1}{n} \right)\]

Hence, by a simple induction $f(1/n) = 0$ if $n > 1$.  On the other
hand, we have
\begin{eqnarray*}
  f(m/n) &=& f\left( \frac{f(\lceil n/m\rceil m/n)}{\lceil n/m\rceil}
  \right) \\
  &=& f\left( \frac{ f\left( 1 + \left(\lceil \frac{n}{m} \rceil -
    \frac{n}{m} \right) \frac{m}{n} \right) }{\lceil n/m\rceil}
  \right) \\
  &=& f\left( \frac{1 + f\left( \frac{m\lceil n/m\rceil - n}{n}
    \right) }{\lceil n/m\rceil}\right)
\end{eqnarray*}

Thus using a strong induction on $m > 0$ we deduce that $f(m/n) = 0$
if $m < n$.  Since
\[ f(x) = f(\{x\}) + \lfloor x\rfloor,\]
we then deduce that $f(x) = \lfloor x\rfloor$ if $x$ is a rational.

On the other hand, if $f(1/2) > 0$.  If we note $g(x) = -f(-x)$, then
\begin{eqnarray*}
  g(x+1) &=& g(x)+1 \\
  g(x) &=& g(g(nx)/n) \\
  g(1/2) &=& 1 + g(1/2 - 1) = 1 - f(1/2) \le 0
\end{eqnarray*}

Hence $g(x) = \lfloor x\rfloor$, thus $f(x) = \lceil x\rceil$.

\end{document}
