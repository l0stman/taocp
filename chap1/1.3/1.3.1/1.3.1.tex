\documentclass[a4paper,12pt]{article}
\newcommand{\newpar}[1]{\bigskip \noindent \textbf{#1.}}
\newcommand{\la}{\leftarrow}
\newcommand{\ra}{\rightarrow}
\newcommand{\subpar}[1]{\medskip \noindent (#1)}
\newcommand{\cell}[1]
           {\begin{tabular}{|*{6}{c|}} \hline #1 \\ \hline \end{tabular}}

\begin{document}
\newpar{1} There would be $4$ ``trits'' per byte.

\newpar{2} We'd need up to $\lceil \log_{64}99999999\rceil = 5$ bytes.

\newpar{3} \subpar{a} (1:2)

\subpar{b} (3:3)

\subpar{c} (4:4)

\subpar{d} (5:5)

\newpar{4} The register I4 can contain an integer value up to
$64 * 64 - 1 = 4096$.  So the effective address in the instruction
could be positive.

\newpar{5} ``LD6 -80,3(0:5)''

\newpar{6}
\subpar{a} $ra \la \cell{-&5&1&200&15}$

\subpar{b} $I2 \la \cell{-&200}$

\subpar{c} $rX \la \cell{+&0&0&5&1&?}$

\subpar{d} undefined

\subpar{e} $rX \la \cell{-&0&0&0&0&0}$

\newpar{7} Let $n = |rAX|$ be the magnitude of the register $rA$ and $rX$
and let $d = |V|$ the magnitude of the divisor.  After the operation,
the magnitude of the register $rA$ is $\lfloor n/d\rfloor$ and the
magnitude of the register $rX$ is $n \bmod{d}$.  $rX$ has the sign of $rA$
before the division and the sign of $rA$ is positive if and only if $rA$
and $V$ have the same sign before the division.

\newpar{8} $rA \la \cell{-&0&617&0&1}$ and $rX \la
\cell{-&0&0&0&1&1}$.

\newpar{9} ADD, SUB, DIV, INCA, INCX, DECA, DECX, JOV, JNOV, NUM.

\newpar{10} CMPA, CMPX, CMPi for $1 \le i \le 6$.

\newpar{11} LD1, LD1N, ENT1, ENN1, INC1, DEC1, MOVE.

\newpar{12} ``INC3 0, 3''

\newpar{13} If the overflow toggle is off, with ``JNOV 1001'' 1001 is
loaded into $rJ$.  ``JNOV 1000'' produces an infite loop whereas ``JOV
1001'' does nothing.

With ``JOV 1000'' the execution time changes.

\newpar{15} $70$, $80$, $120$

\newpar{16}
\subpar{a}
\begin{verbatim}
STZ	0
ENT1	1
MOVE	0(63)
MOVE	0(36)
\end{verbatim}
This program takes $203u$ of time.

\subpar{b}  Use STZ $100$ times and the execution time is equal to
$200u$.

\newpar{17} \subpar{a}
\begin{verbatim}
STZ   0,2
DEC2  1
J2NN  3000
\end{verbatim}

\subpar{b}
\begin{verbatim}
        STZ    0
(3001)  DEC2   63
        J2N    3005
        MOV    0(63)
        JMP    3001
(3005)  INC2   63
(3006)  STZ    0,1
        INC1   1
        DEC2   1
        J2NN   3006
\end{verbatim}

\newpar{18}  The overflow indicator is set on when executing ``ADD
1''.   The comparison indicator is set to \textsc{EQUAL}.  The
register $rI1$ is set to $+3$, the memories $0$ and $2$ are set to
$+0$.  $rA$ contains \cell{-&30&30&30&30&30} and $rX$ contains
\cell{-&31&30&30&30&30}.

\newpar{19} The execution time is $42u$.

\newpage
\newpar{20}
\begin{verbatim}
(0)  ENT1   12
     ENT2   3988     (4000 - 12)
     DEC2   63
     J2N    6
     MOV    11(63)
     JMP    3
(6)  INC2   63
     ST2    8(4:4)
(8)  MOV    11
     ENT1   0
     MOV    11(12)
(11) HLT    0
\end{verbatim}

\newpar{21}
\subpar{a} No.  The J-register is changed only when a jump occurs, and it
contains the address of the next instruction.

\subpar{b}
\begin{verbatim}
          LDA    -1,4
          LDX    3004
          STX    -1,4
          JMP    -1,4
(3004)    JMP    3005
(3005)    STA    -1,4
\end{verbatim}

\newpar{22}  Program using a minimum number of MIX memory locations.
\begin{verbatim}
          ENT1   12
          LDA    2000
(3002)    DEC1   1
          MUL    2000
          SLC    5
          J1NN  3002
          HLT    0
\end{verbatim}
The running time is $171u$ and the space occupied
is $7$.

For the program having the minimum execution time possible, we use the
following property $X^{13} = X
\left(X^2\right)^2\left[\left(X^2\right)^2\right]^2$.   We use $5$
multiplications instead of $12$.

\newpage 
\begin{verbatim}
(3000)   NOP    0
         LDA    2000
         MUL    2000
         STX    3000
         LDA    3000
         MUL    3000
         STX    3000
         LDA    3000
         MUL    3000
         SLC    5
         MUL    3000
         SLC    5
         MUL    2000
         SLC    5
         HLT    0
\end{verbatim}
The running time is $66u$ and the space $15$.

\newpar{23}
Here's one using \textbf{MIX}'s load and store partial fields of
words.
\begin{verbatim}
(3000)  ENT1  5
        ENT2  45           (partial field: 45 = 8 * 5 + 5)
(3002)  ST1   3002(4:4)
        LDA   200
        SLA   1
        DEC1  5
        DEC2  9
        JNN1 3002
\end{verbatim}
Here's one using addition:
\begin{verbatim}
(3000)  NOP   0
        ENT1  5
        LDX   200
(3003)  ENTA  0
        SRC   1
        SLA   0,1
        ADD   3000
        STA   3000
        DEC1  1
        J1NN  3003
\end{verbatim}

\newpar{24}
\begin{verbatim}
SLC    1
SRX    1
\end{verbatim}

\newpar{26}
\begin{verbatim}
00      IN    16(16)      read the second card
01      IN    32(16)      read next card into buf
02      JBUS  2(16)       wait until the transfer is complete
03      LDA   33(1:5)
04      SLA   1
05      SRAX  6           rAX contains the address characters
06      NUM   0           rA <- address
07      STA   11(1:2)
08      LDX   33(1:1)     
09      SRA   5           rA <- 0
10      NUM   1
11      JAZ   0           transfer card
11      STA   31(5:5)     31 <- counter
12      LDA   31(5:5)
13      SUB   10(2:2)
14      STA   31          count <- counter-1
15      LD1   31(5:5)
---------------------------------------------------------------
16      LDA   34,1(5:5)   last byte of the word
17      SUB   21(2:2)
18      LDA   0(0:0)      load the word sign into rA
19      LDA   34,1(1:5)
20      LDX   35,1(1:5)       
21      NUM   30           rA <- counter-th word
22      LDX   11(1:2)      rX <- address of word 1
23      STX   24(1:2)
24      STA   0,1
25      JNN1  12
26      JMP   1
27      NOP   0
28      NOP   0
29      NOP   0
30      NOP   0
31      NOP   0
\end{verbatim}
\end{document}
